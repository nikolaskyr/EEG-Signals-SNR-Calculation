\section{\centerline{Related Work}}
\label{sec:related}


\vspace {15pt}
\subsection{\bf{Cursor Control by Wolpaw et. al.:}} 
The authors in the work in \citep{Wolpaw1991} present a one-dimensional cursor movement BCI, where the EEG were recorded from the scalp at the central sulcus. The task of the subjects was to move vertically a cursor from the centre of a video screen either upwards or downwards. The detection of the subjects’ intent was based on the EEG signals in the mu frequency band.  The work presented in \citep{Wolpaw2004} is a continuation of the work in \citep{Wolpaw1991}. 

The objective of this work was to show that the use of non-invasive BCI can assist people with spinal cord injuries by providing them the capability of real time multidimensional point to point movement control. This study was done with four human subjects, two with disabilities and two with no disabilities, and varying prior experience in using BCI technology. The task of the subjects was to move a cursor appearing in the centre of a computer screen to a target position, place at the edges of the screen. The control of the cursor movement was achieved by examining the amplitudes in the mu and the beta rhythm frequency bands in relation to the EEG signal recorded form the sensorimotor cortex (C3 and C4). Vertical movement control is based on the amplitudes of the C3 and C4 signals at the frequency of 24Hz.  Horizontal movement control is based on the amplitudes of the C3 and C4 signals at the frequency of 12Hz. The results obtained, with respect to the movement time, precision and accuracy are comparable to invasive BCI methods.

\subsection{\bf{The Mu (De)synchronization by Pfurtscheller et. al.:}}

The aim of this work \citep{Pfurtscheller2006} was to examine the effect of the imagination of the hands, foot and tongue movement with the EEG mu frequency band. This study was done with nine able-bodied subjects using 60 EEG electrodes. Through various trials, the effect on specific frequencies in the mu band was characterised with respect to the desynchronization and synchronization observed. The significance of this work was on the conclusion that event related desynchronization (ERD) and event-related synchronization can be used in BCI operated by motor imagery.


\subsection{\bf{A Hybrid BCI System Based on MI and Eye-Blinking by Liu et al.:} }
In this work \citep{Liu2018} the authors propose a hybrid BCI based on motor imagery with the aim to solve the problem of identifying the time at which the concious brain activity starts and ends. The proposed hybrid BCI combines an EEG recording system with an EOG system that provides the activity triggering signals by the state and position of the eyes of the subject. The start of the MI activity is signaled by the used by blinking his eyes. This triggers the EEG system to analyse and determine the intent of the user. In a similar way the user marks the end of the MI activity. 

This hybrid BCI system was tested with a 16 electrode data acquisition system. The task of the subject performing the evaluation experiments was to move a cursor at the screen of a computer following a predifined path. The decision on the subjects motor imagery command was determined according to the changes of the energy levels at the mu and the beta rhythm taken from the cerebral cortex positions.   


\subsection{\bf{BCI for Vehicle Navigation by Babu et. al.:}}
The work presented by the authors in \citep{Babu2017} is similar to the work presented in \citep{Liu2018} above, in the sense that they both use eye blinking to mark the beginning and the end of the motor imagery process. This is achieved by two EEG electrodes attached on the Fp1 and the Fp2 positions to identify eye blinking to mark the beginning or the end of the movement. The proposed BCI is used for the control of the movement of a vehicle. The motor imagery for the navigation of the vehicles is obtained through two electrodes connected on the C3 and C4 positions. The signals used for the control of the direction of the movement are in the mu band. The imagine direction of the movement is determined according to the power of the signals from the C3 and C4 electrodes.
