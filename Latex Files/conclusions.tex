\newpage
\section{\centerline{Conclusions and Future Work}}
\label{sec:conclusions}

\vspace {20pt}

This report presents a study on the ability of a BCI system to detect an EEG signal using a single electrode connected to the Fp1 position. This investigation is concentrated on the alpha band frequencies (8Hz to 12Hz) with special interest on the 10Hz peaks.

One of the main actions taken towards the completion of this work, was the setup of the equipment to be used in order to conduct the experiments and record the EEG data. A number of problems were encountered in the process of setting up the experimental equipment. A consequence of that was a delay in the steps described in the project schedule submitted with the Project Proposal. The first problem was related to the assembly of a cable connecting the amplifier with the data acquisition board. The pin numbers on the amplifier connector given in the schematic diagram were specified wrong. As a consequence of that, the EEG recording system was not functioning, until the problem was identified and fixed. The second issue was related to the incompatibility of the amplifier output voltage with the maximum input voltage of the data acquisition board. The amplifier output voltage swings from -5V to +5V, while the maximum input voltage of the data acquisition board is -1.33V to +1.33V. This incompatibility, in conjunction with the gain of the amplifier resulted in exceeding the maximum input voltage of the analogue to digital converter creating excessive harmonics on the recorded signals. This problem was addressed by changing the gain of the amplifier from 500 to 50.

The methodology designed for this project included the recording of EEG signals from a number of willing persons. The use of people willing to participate as the subject requires a prior approval by the Ethics Committee of the University. An application for the approval of the process has been submitted to the Ethics Committee with a considerable delay due to the hardware problems mentioned above. The approval of this request was not granted till the deadline of the submission of the project report. As a consequence of that, the analysis of the EEG signals was made using EEG data recorded using the author of this report as the only subject.
The recorded EEG signals were investigated with respect of the existence of a peak at the 10Hz frequency and on the calculation of the SNR. To this end, the recorded signals were first filtered to remove the 50Hz mains noise and then filtered again in order to isolate the alpha band frequencies. As far as the 10Hz peak is concerned, it was observed by using the frequency spectrum plots, that the 10Hz was indeed present and that its amplitude can vary according to the motor imagery action of the subject. Other observations on this 10Hz peaks are that (a) these peaks can be shifted to lower or higher frequencies such as 9Hz or 11Hz, and (b) these peaks are not constant during the mental task effort of the subject.
   
The second and main investigation of this project was the calculation of the SNR of the recorded EEG signals, in order to decide whether the proposed BCI setting with a single electrode attached on the Fp1 position can correctly detect the EEG signal. The calculated SNR were then compared with the noise walls determined in a project conducted in 2018.  The calculated SNR were found to be bigger than the SNR walls corresponding to the relaxed SNR wall and the Sudoku solving SNR wall. On the other side, the calculated SNR was in almost all cases smaller than the noise wall of the eye blinking artefact. The conclusion from these observations on the SNR, is that a BCI with a single electrode connected to the Fp1 position can safely decode the EEG signals. However, care must be taken to avoid or reduce the noise due to artefacts.

Future work for the improvement of this work is primarily related to the use of a number of subjects to record EEG signals that will result in more reliable conclusions. A second issue that can improve the accuracy of the results is related to the fact that for the calculation of the SNR, the noise from artefacts was assumed to be zero. This however is not always true, since artefacts can be caused unintentionally. To this end, a de-noising scheme can be included in the methodology in order to remove as much of the artefact noise as possible. 
